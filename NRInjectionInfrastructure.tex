\documentclass[a4paper, 11pt]{article}

\usepackage{epsfig}
\usepackage{graphics}
\usepackage{graphicx}
\usepackage{amsmath,amssymb,mathrsfs}
\usepackage{amsfonts}
\usepackage{color}
\usepackage{wasysym}
\usepackage{times}
\usepackage{mathptmx}
\usepackage{gensymb}
\usepackage{fullpage}
\usepackage{appendix}
\usepackage{fontenc}
\usepackage{listings}

%%%%%%%%%%
\begin{document}
%%%%%%%%%%

\title{Numerical Relativity Injection Infrastructure}
\author{Patricia Schmidt and Ian Harry}
\date{\today}

\maketitle

%%%%%%%%%%%%%%%
\section*{Abstract}
\label{sec:abs}
%%%%%%%%%%%%%%%
This technical document details the standardised infrastructure in the LIGO Algorithms Library (LAL), 
which allows for the usage of Numerical Relativity (NR) waveforms as a discreet waveform approximant
in LIGO data analysis. 

%%%%%%%%%%%%%%%
\section{Introduction}
\label{sec:intro}
%%%%%%%%%%%%%%%
Gravitational waveforms from coalescing compact binary systems are commonly used in LIGO data analysis.
Whilst for low-mass systems only the early part of the inspiral is accessible to LIGO, for high-mass systems 
also the later stages of the binary evolution, in particular the merger and final ringdown, may enter the LIGO
sensitivity band. The waveforms in the early part of the binary coalescence are accurately described by 
analytic post-Newtonian of the Einstein field equations, but the final stages require the full non-linear solution
of the field equations. So far, NR merger waveforms have been used to calibrate analytic waveform models,
and to test those models against independent NR waveforms not used in the construction. 

However, in the advanced detector era, we may wish to directly use NR waveforms in gravitational-wave searches,
parameter estimation, to test General Relativity and to assess the systematics of analytic waveforms models within
a uniform framework. To do so, we have recently developed a simple infrastructure to allow for the treatment of
NR waveforms as a ``discreet'' waveform approximant. 

Previously, hybrid waveforms constructed by combining a PN inspiral with an NR merger-ringdown waveform have been
used in LIGO data analysis and parameter estimation in the NINJA and NINJA-2 projects. The new injection infrastructure
supersedes the previous method.

The remainder of this technical document is organised as follows: In Sec. \ref{sec:format} we give a brief introduction
to the format of the NR waveform data. In Sec. \ref{sec:gen} we describe the code basics and how NR waveforms
are generated in LAL/pyCBC.

%%%%%%%%%%%%%%%
\section{Waveform Format}
\label{sec:format}
%%%%%%%%%%%%%%%
In a Numerical Relativity simulation, one solves for the complete space-time of the binary system. For data analysis
purposes however, one requires the gravitational-wave strain $h(t)$ far from the source. The relevant numerical quantity
is the metric perturbation $h^{TT}_{ij}$ as computed in the transverse-traceless gauge (TT). 

There are different way of computing the metric perturbation from a numerical evolution. The most common methods include
the use of the complex Weyl scalar $\Psi_4$, which is related to the metric perturbation via two time derivatives, or the
Regge-Wheeler-Zerilli formalism computes the metric perturbation in the wave-zone as a perturbation of the Schwarzschild
spacetime. 

In the TT gauge, the metric perturbation
has two independent real polarisations, $h_+$ and $h_\times$, which are related to the complex strain by
\begin{equation}
\label{ }
h = h_+ - i h_\times \in \mathbb{C},
\end{equation}
where $h_+, h_\times \in \mathbb{R}$.

Let $(t,x,y,z)$ be a Cartesian coordinate system in the wave-zone, which is related to the polar coordinates $(r, \theta, \phi)$ by the standard means. 
In this coordinate system, the metric perturbation is commonly decomposed into modes
in a basis of spin-weighted spherical harmonics, ${}^{-2}Y_{\ell m}$, of spin weight $s=-2$, where the GW propagation direction is
the radial unit vector r.
For any point $(\theta, \phi)$ on the unit sphere, the GW strain can then be shown to take the following form:
\begin{equation}
\label{ }
h(t; \theta, \phi) = h_+ - i h_\times = \sum_{\ell=2}^\infty \sum_{m=-\ell}^{\ell} h_{\ell m}(t) {}^{-2}Y_{\ell m}(\theta,\phi).
\end{equation}

As for any wave, we can also write each gravitational-wave mode $h_{\ell m}$ as an amplitude $A_{\ell m}$ and a phase
$\Phi_{\ell m}$,
\begin{equation}
\label{ }
h_{\ell m}(t) = A_{\ell m}(t)e^{i\Phi_{\ell m}}.
\end{equation}
The amplitude of each mode is its complex norm, the phase is unwrapped argument of the complex time series $h_{\ell m}$. 

%%%%%%%%%%%%%%%
\subsection{Spline Compression}
\label{sec:spline}
%%%%%%%%%%%%%%%
Gravitational waveforms for LIGO data analysis purposes require uniform sampling in time for a given sampling frequency $f_s$.
NR datasets however, are commonly not uniformly sampled and if they are, the sampling interval $dt$ may not necessarily
correspond to the one required by the sampling rate. It is therefore unavoidable to interpolate the NR data such that for any total
mass M and sampling rate the NR waveforms can be resampled accordingly. Whilst the NR data could simply be interpolated as
they are, we choose the reduce the data load and therefore also the I/O time by performing 1D spline compression on the NR data
\cite{}. This is a particular advantage for long simulations or hybrid data. \\

The 1D spline compression is performed separately for each amplitude $A_{\ell m}$ and phase $\Phi_{\ell m}$. The compression is
achieved by applying a greedy algorithm to the NR data, which selects the near-optimal points to construct a univariate spline interpolant
with a specified global accuracy. By default, the interpolants are constructed using fifth degree polynomials and a tolerance of $10^{-6}$,
i.e. if the spline is evaluated at the discreet times $t_i$ of the NR data, the original NR values are recovered with an error smaller than the
specified tolerance. For further details of this method and the accuracy of the obtained interpolants, we refer the reader to \cite{}. 
To perform the spline compression, the publicly available python package ``romSpline'' by Chad R. Galley is used \cite{}. \\

Once the spline interpolants have been obtained, they are stored as individual groups in a single HDF5 file with the following group naming
conventions: 
\begin{itemize}
  \item Amplitude group: \textbf{amp\textunderscore l\#1\textunderscore m\#2}
  \item Phase group: \textbf{phase\textunderscore  l\#1\textunderscore m\#2},
\end{itemize}
where (\#1, \#2) are placeholders for $(\ell, m)$, e.g. phase\textunderscore l2\textunderscore m-2. \\
\\We note that for NR data without a hybridised PN inspiral, the initial junk radiation needs to be removed before the spline interpolants are constructed. 
Following the LAL waveform convention, the waveforms also need to be aligned such that the peak of the waveform occurs at $t_\mathrm{peak}=0$ 
before constructing the interpolants.

%%%%%%%%%%%%%%%
\subsection{Metadata}
\label{sec:meta}
%%%%%%%%%%%%%%%
Each HDF5 has to include a minimal set of metadata associated with the NR simulation stored as attributes of the HDF5 file. 
More metadata can be added as desired but the minimal requirement are the following: \\
\newline
{[}NR-group{]}: a string to identify the NR group/code \\
{[}type{]}: a brief description of the simulation, e.g. aligned-spins, precessing etc. \\
{[}name{]}: list here alternative names or internal identifiers of the simulation \\
{[}eta{]}: the symmetric mass ratio of the binary system \\
{[}f\textunderscore lower\textunderscore at\textunderscore 1MSUN{]}: frequency of the $(2,2)$-mode in Hz at the beginning of the waveform scaled to 1 solar mass \\
{[}coa\textunderscore phase{]}: this is twice the orbital reference phase at the beginning of the waveform \\
{[}spin1x{]}, {[}spin1y{]}, {[}spin1z{]}: spin components of the spin on the larger object measured w.r.t. to the orbital angular momentum L at the beginning of the waveform \\
{[}spin2x{]}, {[}spin2y{]}, {[}spin2z{]}: spin components of the spin on the smaller object measured w.r.t. to the orbital angular momentum L at the beginning of the waveform \\
\\ To reduce potential ambiguities and to enable the usage of waveforms from matter simulations, we further recommend adding the following metadata: \\
{[}object1{]}, {[}object2{]}: a string to identify the object type, e.g. black hole, neutron star \\
{[}mass1{]}, {[}mass2{]}: matter simulations have a mass scale, hence the not only the mass ratio but also the individual masses need to be accessible\\
{[}eccentricity{]}: eccentricity parameter of the simulation\\
{[}NR\textunderscore frame{]}: the coordinate system used for the waveform extraction, e.g. inertial, J-aligned etc. \\
{[}J\textunderscore hat{]}: the direction of the total angular momentum at the beginning of the waveform\\
{[}L\textunderscore hat{]}: the direction of the orbital angular momentum at the beginning of the waveform \\
{[}PN\textunderscore approximant{]}: string identifier for the inspiral if hybrid data are stored \\
\\ The spin measurements and the orbital phase at the beginning of the waveform have to be measured in a frame where the instantaneous orbital plane
lies in the xy-plane and $\hat{L}$ points along the z-axis. 

For complete reproducibility and transparency, we recommend to store an NR metadata files and the original NR times $t_i$ as datasets in the HDF5 file.

%%%%%%%%%%%%%%%
\section{Generating NR waveforms}
\label{sec:gen}
%%%%%%%%%%%%%%%
Once the HDF5 file has been provided, the NR waveforms can be generated through the standard waveform interfaces \texttt{ChooseTDWaveform()} in LAL
and \texttt{get\textunderscore td\textunderscore waveform()} in pyCBC. The approximant names are ``NR\textunderscore hdf5'' and ``NR\textunderscore hdf5\textunderscore pycbc'' respectively. 
As opposed to continuous approximants, the spline data are read from file and evaluated for the desired extrinsic
parameters. The intrinsic parameters in NR simulations are fixed, currently internal checks on the mass ratio and the spin components 
are performed to guarantee the consistency between the values passed
in the waveform generation call and metadata values. \\
\\For a given starting frequency and total mass, a time array is allocated based on an estimate of the waveform length. We use the LAL function 
\texttt{SimIMRSEOBNRv2ChirpTimeSingleSpin()} plus an extra 10\% leverage to estimate the waveform length. If the NR waveforms are not long
enough, the generation is aborted. From the estimated length and the desired sampling rate, the discreet time series for the spline evaluation is
determined.

To construct the polarisation $h_+$ and $h_\times$, the splines for each amplitude and phase are evaluated at the points in the discreet time times, 
weighted by the ${}^{-2}Y_{lm}$, and summed up:
\begin{align}
\label{}
    \mathrm{Re}(h_{\ell m}) &= A_{\ell m} \cos(\Phi_{\ell m}),   \\
    \mathrm{Im}(h_{\ell m}) &= A_{\ell m} \sin(\Phi_{\ell m}),   \\
    h_+ &= \sum_{\ell, m} \mathrm{Re}(h) \mathrm{Re}({}^{-2}Y_{\ell m}) + \mathrm{Im}(h) \mathrm{Im}({}^{-2}Y_{\ell m}), \\
    h_\times &= \sum_{\ell, m} \mathrm{Re}(h) \mathrm{Im}({}^{-2}Y_{\ell m}) - \mathrm{Im}(h) \mathrm{Re}({}^{-2}Y_{\ell m}),
\end{align}
where the polar angle $\theta$ denotes the angle between the orbital angular momentum and the line-of-sight and the azimuthal
angle $\phi$ is the sum of parameter value 'coa\textunderscore phase' and any reference phase passed in the waveform generator function.
\\
Note that all waveform modes $(\ell, m)$ contained in the HDF5 file are used
to compute the two polarisations. \\
\\
The compressed NR data files do not store the splines themselves, but the knots, errors, the polynomial degree etc. In the pyCBC implementation,
the splines are constructed from the HDF5 file via the python function \texttt{UnivariateSpline}, in the LAL implementation regular GSL interpolation
is performed. Fig. XX shows a comparison between NR waveforms obtained using the the pyCBC and the LAL waveform generators. The mismatch
between the waveforms is less than $10^{-7}$. This is consistent with the level of disagreement expected due to the different numerical interpolation
routines. \\
The source codes can be found in lalsuite/lalsimulation/src/LALSimIMRNRWaveforms.c and 
pycbc/pycbc/waveforms/nr\textunderscore waveform.py respectively.

%%%%%%%%%%
\subsection{Examples}
%%%%%%%%%%
The pyCBC implementation of the waveform generator may be called in the following way: \\
\begin{lstlisting}
from pycbc.waveform import get_td_waveform 
hp, hc = get_td_waveform(approximant='NR_hdf5_pycbc', 
                         numrel_data='/PATH/TO/HDF5',
                         mass1, mass2,
                         spin1x, spin1y, spin1z,
                         spin2x, spin2y, spin2z, 
                         delta_t, f_lower, f_ref,
                         inclination, distance, 
                         coa_phase)
\end{lstlisting}
To use the implementation in LALSimulation through SWIG, use the following function call:\\
\begin{lstlisting}
import lalsimulation as lalsim
flags = lalsim.SimInspiralCreateWaveformFlags()
lalsim.SimInspiralSetNumrelData(flags,'/PATH/TO/HDF5')
hp, hc = lalsim.SimInspiralChooseTDWaveform(phiRef, 
              deltaT, m1 * MSUN_SI, m2 * MSUN_SI, 
              s1x, s1y, s1z,
              s2x, s2y, s2z, 
              fStart, fRef, 
              distance, inclination, 
              0, 0, flags, None, -1, -1, 
              approximant=lalsim.NR_hdf5)
\end{lstlisting}
%%%%%%%%%%
%\appendix{Conventions and frame choices}
%%%%%%%%%%


%%%%%%%%%%
\end{document}
%%%%%%%%%%
